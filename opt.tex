\documentclass[letterpaper, 10pt]{amsart}
\usepackage[utf8]{inputenc}

\title{Optimizing the Daban Urnud}
\author{David Prentiss}
\date{July 2016}

\begin{document}


\maketitle

\section{Water Level}

\begin{quote}
Blow a balloon out of steel, almost a mile wide, and fill it half full of water. Repeat three more times. Place these four orbs at the corners of a square, close to one another, but not quite touching.

...

The orbs themselves are comparatively simple. Inside of them, the water's free to find its own level. When the whole construct is spinning, the water flees to the outside and settles into a curve on which 'gravity' is always equal to what it was on the home planet. When the ship is under power, the water settles into the aft part of the sphere and levels out. 

--- Neal Stephenson, \em{Anathem}
\end{quote}



According to Saunt Stephenson, orbs on the Daban Urnud are half full of water.
We will take some liberty with the quantity ``half full" to optimize the water level for various objectives that may be of interest to Geometers. 

\subsection{Surface Area}

Consider an orb in the Daban Urnud with diameter $r_o$.
Assume for our purposes that is perfectly spherical and ignore the thickness of the walls and any gaps between spheres.
The centroid of the orb is coplanar with that of the three other orbs arranged in simple-cubic packing such that the centroids comprise a square with sides of length $2r_o$. 
Assume the rotational axis of the Daban Urnud's orbs is orthogonal to that plane and passes through the centroid of the implied square.
As such, the centroid of the orb will sweep out a circle with radius equal to its distance from the axis or $\sqrt{2}r_o$.
The fluid inside an orb will be accelerated away from the axis of rotation and form a radial pressure gradient.
Assuming the orb is filled with mostly water and air that are free to move, the more dense water will fill the higher-pressure extents of the gradient to form a cylindrical surface.


\[
    \theta = \arccos \left(\frac{r_o^2+r_w^2}{2 \sqrt{2} r_o r_w}\right)
    \]
\[
    \ell = 2 r_w\arccos \left(\frac{r_o^2+r_w^2}{2 \sqrt{2} r_o r_w}\right)
    \]
\[
    \sqrt{r_o^2-x^2} = \sqrt{2}r_o - r_w
    \]
\[
    A = 4 r_w \int_0^{\sqrt{-r_o^2+2\sqrt{2}r_o r_w-r_w^2}} \arccos \left(\frac{\sqrt{r_o^2-x^2}+r_w^2}{2 \sqrt{2} r_o r_w}\right) dx
    \]
\[
    \max_{r_w} A = 4 r_w \int_0^{\sqrt{-r_o^2+2\sqrt{2}r_o r_w-r_w^2}} \arccos \left(\frac{\sqrt{r_o^2-x^2}+r_w^2}{2 \sqrt{2} r_o r_w}\right) dx
    \]
    
\end{document}
